\documentclass[a4paper, 14pt]{extarticle}
\usepackage[russian]{babel}
\usepackage[T1]{fontenc}
\usepackage{fontspec}
\usepackage{indentfirst}
\usepackage{enumitem}
\usepackage{graphicx}
\usepackage{mathtools}
\usepackage[
  left=20mm,
  right=10mm,
  top=20mm,
  bottom=20mm
]{geometry}
\usepackage{parskip}
\usepackage{titlesec}
\usepackage{xurl}
\usepackage{hyperref}
\usepackage{float}
\usepackage[
  figurename=Figure,
  labelsep=endash,
]{caption}
\usepackage[outputdir=build, newfloat]{minted}

\hypersetup{
  colorlinks=true,
  linkcolor=blue,
  filecolor=blue,
  urlcolor=blue,
}

\renewcommand*{\labelitemi}{---}

\newenvironment{code}{\captionsetup{type=listing}}{}
\SetupFloatingEnvironment{listing}{name=Листинг}

\setminted{
  fontsize=\footnotesize,
}

\setmonofont{JetBrains Mono}[
  SizeFeatures={Size=11},
]

\setlength{\parskip}{6pt}

\setlength{\parindent}{1cm}
\setlist[itemize]{itemsep=0em,topsep=0em,parsep=0em,partopsep=0em,leftmargin=2.0cm}
\setlist[enumerate]{itemsep=0em,topsep=0em,parsep=0em,partopsep=0em,leftmargin=2.0cm}

\renewcommand{\thesection}{\arabic{section}.}
\renewcommand{\thesubsection}{\thesection\arabic{subsection}.}
\renewcommand{\thesubsubsection}{\thesubsection\arabic{subsubsection}.}

\titleformat{\section}{\normalfont\bfseries}{\thesection}{0.5em}{}
\titleformat{\subsection}{\normalfont\bfseries}{\thesubsection}{0.5em}{}

\titleformat*{\section}{\normalfont\bfseries}
\titleformat*{\subsection}{\normalfont\bfseries}

\linespread{1.5}
\renewcommand{\baselinestretch}{1.5}
\begin{document}

\begin{flushright}
  K33211 Daniil Shvalov
\end{flushright}

\begin{center}
  \textbf{Homework 4}

  Banking System DBMS
\end{center}

Today we will look at the ERD diagram for the banking system. In this diagram
there are 4 entities.
\begin{itemize}
  \item The first entity is an Account that has attributes such as account number and
  balance. The account number is the primary attribute.
  \item The next entity is a Branch that has attributes such as branch name,
  city and assets. The branch name is the primary attribute.
  \item Another entity Loan contains a loan number and amount. The loan number
  is the primary attribute.
  \item The last entity is a Customer, which has attributes such as customer
  name, city and street. The customer name is the primary attribute.
\end{itemize}

Consider the relationship between entities.
\begin{itemize}
  \item Each Loan must belong to one and only one Branch. On the other hand,
  Branch can have a lot of Loans. At the same time, Branch may not have Loans at
  all. Loan determines its relation to Branch using branch name attribute. The
  cardinality is one-to-many.
  \item Each Account must belong to one and only one Branch. Branch can also
  have many accounts. It may also be that no Account belongs to Branch. Account
  determines its relationship to Branch using branch name attribute. The
  cardinality is one-to-many.
  \item There is a many-to-many relationship between Account and Customer
  entities through the account number attribute and the customer name attribute.
  \item Between the entities Customer and Loan, the relationship has a
  cardinality of many-to-many using the attribute of the customer name and the
  attribute of the loan number.
\end{itemize}

Now let's look at the representation of our entities in a relational model:
\begin{itemize}
  \item Account entity has the attributes account number, branch name and
  balance. The branch name is a foreign key that is connected to the branch name
  in the entity Branch.
  \item Branch entity has the attributes branch name, city and assets.
  \item Loan entity has the attributes loan number, branch name and amount. The
  branch name is a foreign key that is connected to the branch name in the
  entity Branch.
  \item Customer entity has the attributes customer name, city and street.
\end{itemize}

Also, to implement many-to-many relationships, two additional
entities are introduced: Depositor and Borrower.
\begin{itemize}
  \item Depositor has the attributes of the customer name and account number.
  The customer name is a foreign key that is connected to the attribute of the
  customer name of the Customer entity. The account number is a foreign key that
  is connected to the account number attribute of the Account entity.
  \item Borrower has the attributes of the customer name and loan number. The
  customer name is a foreign key that is connected to the attribute of the
  customer name of the Customer entity. The loan number is a foreign key that is
  connected to the loan number attribute of the Loan entity.
\end{itemize}

This is all that can be said about this schema. I hope that it has become a
little clearer to you how the banking system works from the inside at the
database level.

\end{document}
