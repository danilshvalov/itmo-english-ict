\documentclass[a4paper, 14pt]{extarticle}
\usepackage[russian]{babel}
\usepackage[T1]{fontenc}
\usepackage{fontspec}
\usepackage{indentfirst}
\usepackage{enumitem}
\usepackage{graphicx}
\usepackage{mathtools}
\usepackage[
  left=20mm,
  right=10mm,
  top=20mm,
  bottom=20mm
]{geometry}
\usepackage{parskip}
\usepackage{titlesec}
\usepackage{xurl}
\usepackage{hyperref}
\usepackage{float}
\usepackage[
  figurename=Figure,
  labelsep=endash,
]{caption}
\usepackage[outputdir=build, newfloat]{minted}

\hypersetup{
  colorlinks=true,
  linkcolor=blue,
  filecolor=blue,
  urlcolor=blue,
}

\renewcommand*{\labelitemi}{---}

\newenvironment{code}{\captionsetup{type=listing}}{}
\SetupFloatingEnvironment{listing}{name=Листинг}

\setminted{
  fontsize=\footnotesize,
  frame=lines,
  framesep=2mm,
}

\setlength{\parskip}{6pt}

\setlength{\parindent}{1cm}
\setlist[itemize]{itemsep=0em,topsep=0em,parsep=0em,partopsep=0em,leftmargin=2.0cm}
\setlist[enumerate]{itemsep=0em,topsep=0em,parsep=0em,partopsep=0em,leftmargin=2.0cm}

\renewcommand{\thesection}{\arabic{section}.}
\renewcommand{\thesubsection}{\thesection\arabic{subsection}.}
\renewcommand{\thesubsubsection}{\thesubsection\arabic{subsubsection}.}

\titleformat{\section}{\normalfont\bfseries}{\thesection}{0.5em}{}
\titleformat{\subsection}{\normalfont\bfseries}{\thesubsection}{0.5em}{}

\titleformat*{\section}{\normalfont\bfseries}
\titleformat*{\subsection}{\normalfont\bfseries}

\linespread{1.5}
\renewcommand{\baselinestretch}{1.5}
\begin{document}

\begin{flushright}
  K33211 Daniil Shvalov
\end{flushright}

\begin{center}
  \textbf{Homework 1}
\end{center}

This task was taken from the Mathematics for Computer Science course of the MIT,
Problem Set 4, Problem 2 (see
\href{https://ocw.mit.edu/courses/6-042j-mathematics-for-computer-science-fall-2010/resources/mit6_042jf10_assn04/}{https://ocw.mit.edu/courses/6-042j-mathematics-for-computer-science-fall-2010/resources/mit6\_042jf10\_assn04/}).

\textbf{Description}. Let \(G = (V, E)\) be a graph. Recall that the degree of a
vertex \(v \in V\), denoted \(d_v\), is the number of vertices \(w\) such that
there is an edge between \(v\) and \(w\).

\textbf{Problem 1}. Prove that
\[
  2|E| = \sum_{v \in V} d_v.
\]

\textbf{Solution}. Let's take a graph in which no vertex is connected to another
edge. The sum of the degrees of the vertices of this is zero. When adding an
edge connecting any two vertices, the sum of all degrees increases by 2. Thus,
the sum of all degrees of vertices is even and equal to twice the number of
edges.

\textbf{Problem 2}. At a 6.042 ice cream study session (where the ice cream is
plentiful and it helps you study too) 111 students showed up. During the
session, some students shook hands with each other (everybody being happy and
content with the ice-cream and all). Turns out that the University of Chicago
did another spectacular study here, and counted that each student shook hands
with exactly 17 other students. Can you debunk this too?

\textbf{Solution}. In this case, \(|V| = 111\), \(|E| = 17\). It follows from what
was proved earlier that the sum of the degrees of the vertices is always even.
In this case, since each student shook hands exactly 17 times, the sum of the
degrees is equal to
\[
  \sum_{v \in V} d_v = |V| \cdot |E| = 111 \cdot 17 = 1887.
\]
We got an odd number, which means that such a graph cannot exist and the
University of Chicago was wrong.

\textbf{Problem 3}. And on a more dull note, how many edges does \(K_n\), the
complete graph on \(n\) vertices, have?

\textbf{Solution}. In a complete graph, every vertex is connected to every other
vertex. This means that each vertex has degree \(d_v = n - 1\). From the
previously proved we have:
\[
  2 |E| = \sum_{v \in V} d_v = n \cdot (n - 1)
  \implies
  |E| = \frac{n \cdot (n - 1)}{2}.
\]

\end{document}
