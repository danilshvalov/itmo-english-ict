\documentclass[a4paper, 14pt]{extarticle}
\usepackage[russian]{babel}
\usepackage[T1]{fontenc}
\usepackage{fontspec}
\usepackage{indentfirst}
\usepackage{enumitem}
\usepackage{graphicx}
\usepackage{mathtools}
\usepackage[
  left=20mm,
  right=10mm,
  top=20mm,
  bottom=20mm
]{geometry}
\usepackage{parskip}
\usepackage{titlesec}
\usepackage{xurl}
\usepackage{hyperref}
\usepackage{float}
\usepackage[
  figurename=Figure,
  labelsep=endash,
]{caption}
\usepackage[outputdir=build, newfloat]{minted}

\hypersetup{
  colorlinks=true,
  linkcolor=blue,
  filecolor=blue,
  urlcolor=blue,
}

\renewcommand*{\labelitemi}{---}

\newenvironment{code}{\captionsetup{type=listing}}{}
\SetupFloatingEnvironment{listing}{name=Листинг}

\setminted{
  fontsize=\footnotesize,
}

\setmonofont{JetBrains Mono}[
  SizeFeatures={Size=11},
]

\setlength{\parskip}{6pt}

\setlength{\parindent}{1cm}
\setlist[itemize]{itemsep=0em,topsep=0em,parsep=0em,partopsep=0em,leftmargin=2.0cm}
\setlist[enumerate]{itemsep=0em,topsep=0em,parsep=0em,partopsep=0em,leftmargin=2.0cm}

\renewcommand{\thesection}{\arabic{section}.}
\renewcommand{\thesubsection}{\thesection\arabic{subsection}.}
\renewcommand{\thesubsubsection}{\thesubsection\arabic{subsubsection}.}

\titleformat{\section}{\normalfont\bfseries}{\thesection}{0.5em}{}
\titleformat{\subsection}{\normalfont\bfseries}{\thesubsection}{0.5em}{}

\titleformat*{\section}{\normalfont\bfseries}
\titleformat*{\subsection}{\normalfont\bfseries}

\linespread{1.5}
\renewcommand{\baselinestretch}{1.5}
\begin{document}

\begin{flushright}
  K33211 Daniil Shvalov
\end{flushright}

\begin{center}
  \textbf{Homework 3}

  C++ programming language
\end{center}

C++ is a compiled, statically typed general-purpose language. What does it mean?
Let's start in order.

To run a program in C++, compilation of the code is required. Without this, you
will not be able to execute the program. Thanks to compilation, C++ programs are
efficient and optimized, both in time and in memory.

Static typing means that all objects that the programmer works with must be
spelled out as language types. Otherwise, the program will not compile. This
helps to catch a lot of errors even at the compilation stage.

C++ is a general purpose language. This means that it can be used for a wide
range of tasks. Basically, C++ is used for high-load systems, systems requiring
low latency, real-time systems, games. If you need to write a web server that
will be able to withstand up to several thousand requests per second, then C++
is your choice.

C++ supports various programming paradigms. For example, C++ supports all the
concepts of object-oriented programming. It also allows you to write code in an
imperative and functional style. C++ is a very flexible language, it practically
does not limit the programmer in the style of writing code and in the approaches
used. This allows you to implement many different programs, while not losing
performance.

The syntax is quite simple and is found in many languages. When declaring a
variable, its type is first specified, then its name, and then its value. When
declaring a function, the type of the return value is first specified, then its
name and arguments. All assignment, initialization, and function call
expressions must end with a semicolon.

Consider an example. Let's say we want to write a function that adds one to
the passed value, that is, an increment function. To do this, first we need to
write the type of the return value. Let it be int. Then we need to write the
name of the function, let's call it 'increment'. After that, we need to write
the arguments of the function. Arguments start with an opening bracket. Then the
type and name are specified for each argument. Arguments are separated by
commas. For the increment function, it is enough for us to have one argument.
We will also choose int as the type and name it as 'arg'. After specifying all
the arguments, there is a closing bracket. Then the body of the function is in
curly brackets. For the increment function, you need to assign the value 'arg'
plus 1 to 'arg', and then return this value using the keyword 'return'.

The 'main' function acts as the entry point to the program in C++. It is called
first when the program starts. The std::cin and std::cout functions are used for
input and output. With the help of double angle brackets, arguments are passed
that need to be output or read into. So, for the increment function, we have to
write std::cout, two left angle brackets, word 'increment', the value to be
incriminated in brackets and a semicolon.

\begin{minted}{cpp}
  int increment(int arg) {
    arg = arg + 1;
    return arg;
  }

  int main() {
    std::cout << increment(1) << '\n';
  }
\end{minted}

C++ is a worthy candidate as a programming language of choice for many tasks. I
believe that in most cases C++ will look advantageous against the background of
other programming languages.

\end{document}
